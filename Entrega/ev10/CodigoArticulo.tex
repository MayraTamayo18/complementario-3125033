\documentclass[twocolumn]{article}

\usepackage[utf8]{inputenc}
\usepackage[spanish]{babel}
\usepackage{hyperref}
\usepackage{geometry}

\geometry{margin=1in}

\title{Enfoques y Herramientas para el Diseño y Mejora de Arquitecturas
de Software en Aplicaciones Web y Otros Contextos Tecnológicos}

\author{Mayra Alejandra Tamayo Perdomo \\
Neiva, Huila \\
\texttt{mairatp2005@gmail.com} \\
Centro de la Industria, la Empresa y los Servicios}

\begin{document}

\maketitle

\section{Introducción}
La arquitectura de software es fundamental en el desarrollo de
aplicaciones, ya que determina la estructura y organización del sistema,
influyendo en su implementación y sostenibilidad. Este artículo revisa
un conjunto de 34 investigaciones que analizan diversas perspectivas
sobre cómo los patrones de diseño impactan en la arquitectura de
software, especialmente en el contexto de aplicaciones web y entornos de
computación en la nube.

\section{Palabras clave:} Arquitectura de software,Patrones de diseño, Desarrollo de software, Usabilidad,Experiencia del usuario,Computación en la nube, Aplicaciones web, Robótica, Ingeniería de software, Calidad del software, Escalabilidad, Mantenibilidad, Arquitectura orientada a servicios (SOA), Generadores de código fuente, Modelado y verificación de patrones de diseño.

\section{Resumen:}

La arquitectura de software y los patrones de diseño son conceptos esenciales en el desarrollo de sistemas que buscan ser robustos, escalables y mantenibles. En un entorno tecnológico en constante evolución, donde la complejidad de las aplicaciones web y los entornos de computación en la nube aumenta, es fundamental contar con metodologías y herramientas que faciliten no solo el aprendizaje, sino también la aplicación efectiva de estos principios.

Este conjunto de artículos reúne una variedad de enfoques y estudios sobre la arquitectura de software y sus patrones, destacando su importancia en el desarrollo contemporáneo.
Uno de los temas más significativos abordados es el modelado y verificación de patrones de diseño en arquitectura de software, especialmente en entornos de computación en la nube. Este enfoque integral permite a los arquitectos de software utilizar arquitecturas genéricas que proporcionan estructuras comunes para resolver problemas específicos. La introducción de un metamodelo y herramientas gráficas para la instanciación de componentes arquitectónicos es una contribución notable, ya que no solo facilita el trabajo de los arquitectos, sino que también asegura la calidad del software mediante la verificación de patrones de diseño. 

Esto es especialmente relevante en un contexto donde la demanda de aplicaciones web escalables y eficientes está en aumento, lo que requiere un enfoque más sistemático y organizado.

Asimismo, varios artículos se centran en la creación de herramientas educativas destinadas a enseñar patrones de diseño de software. Estas herramientas son cruciales para ayudar a los estudiantes a entender conceptos complejos y aplicarlos en contextos prácticos. Al enfocarse en patrones fundamentales, como el Model-View-Controller (MVC), se busca preparar a los alumnos para el mercado laboral, facilitando su integración en el mundo profesional y mejorando su comprensión de las mejores prácticas en el desarrollo de software. La educación continua en este campo es esencial, ya que la falta de conocimiento puede limitar la capacidad de los desarrolladores para aplicar patrones de diseño de manera efectiva. Esto, a su vez, puede afectar negativamente la calidad del software producido.

El uso de patrones de diseño también se extiende al desarrollo de aplicaciones web, donde se destaca la creciente complejidad debido a las demandas de los usuarios. La revisión de patrones aplicados a aplicaciones web subraya la importancia de adoptar buenas prácticas en el desarrollo de software, lo que no solo mejora la usabilidad y el mantenimiento, sino que también proporciona un marco claro para los equipos de desarrollo. Un caso de estudio sobre un sistema de seguimiento que no utilizó patrones de diseño resalta las deficiencias que pueden surgir, lo que refuerza la necesidad de implementar estructuras adecuadas desde el principio. Esta práctica no solo optimiza el desarrollo, sino que también contribuye a la satisfacción del usuario final, un aspecto cada vez más crítico en un mercado competitivo.

Otro enfoque relevante es la arquitectura orientada a servicios (SOA), la cual se discute en el contexto del desarrollo de aplicaciones empresariales. Esta arquitectura permite una integración y comunicación más flexibles entre sistemas, aunque presenta desafíos en la gestión de múltiples conexiones. La propuesta de diseñar servicios autónomos que se comuniquen de manera eficiente es clave para mitigar estos problemas, así como el establecimiento de políticas de seguridad y manejo de excepciones que garanticen la integridad del sistema. En este sentido, se enfatiza la necesidad de desarrollar estrategias que aseguren la robustez y resiliencia de las aplicaciones ante fallos operativos, lo que es fundamental para mantener la confianza del usuario.
La implementación de arquitecturas de software para robots móviles, como es el caso del robot Lázaro, ilustra la aplicación práctica de los patrones de diseño en entornos reales.

Este artículo detalla una arquitectura de tres niveles que optimiza el control y la operación del robot, utilizando diversos sensores para la detección de obstáculos y la medición de fuerzas. La capacidad de comunicación remota y la combinación de arquitecturas deliberativas y reactivas demuestran un enfoque innovador que podría ser aplicable en futuros proyectos de robótica, abriendo nuevas posibilidades en el campo de la inteligencia artificial y la automatización. Estos avances no solo mejoran la funcionalidad del robot, sino que también proporcionan un marco para explorar nuevas aplicaciones en diversos sectores, desde la manufactura hasta la asistencia personal.

La revisión sistemática sobre generadores de código fuente y patrones arquitectónicos destaca la importancia de estas herramientas en el desarrollo de software. Los generadores de código permiten automatizar tareas repetitivas, lo que no solo ahorra tiempo, sino que también reduce la posibilidad de errores, especialmente en proyectos de gran envergadura. Al analizar las ventajas y desventajas de utilizar generadores de código, el estudio proporciona una perspectiva valiosa para investigadores y profesionales en la ingeniería de software, promoviendo prácticas que optimizan la eficiencia del desarrollo. La automatización en la generación de código es un paso crucial hacia el desarrollo ágil, permitiendo a los equipos enfocarse en aspectos más creativos y estratégicos del proceso de desarrollo.
En cuanto a la importancia de la usabilidad, se presenta un enfoque que integra patrones de usabilidad desde el inicio del desarrollo del software. Este planteamiento busca anticipar las necesidades del usuario, minimizando la cantidad de cambios necesarios en etapas posteriores y, en última instancia, mejorando la experiencia del usuario. Al demostrar que la usabilidad es una consideración crítica desde la fase arquitectónica, se busca fomentar un desarrollo más centrado en el usuario y en sus expectativas, lo que resulta en aplicaciones más funcionales y satisfactorias. La atención a la usabilidad no solo mejora la satisfacción del usuario, sino que también puede reducir costos a largo plazo al disminuir la necesidad de revisiones y ajustes posteriores.

Los patrones de diseño GOF (Gang of Four) también son objeto de estudio, y se revela su prevalencia en la industria, aunque se identifican barreras en su adopción debido a la falta de conocimiento y experiencia entre los desarrolladores. La investigación sugiere que, a pesar de que algunos patrones son ampliamente utilizados, como Singleton y Factory Method, es esencial fomentar una cultura de calidad más fuerte en el desarrollo de software para maximizar el uso de estos patrones. La creación de catálogos accesibles y comprensibles sobre estos patrones podría facilitar su aprendizaje y aplicación en proyectos reales, ayudando a cerrar la brecha entre la teoría y la práctica en el desarrollo de software.

Finalmente, el análisis comparativo de patrones de diseño proporciona un recurso valioso para los desarrolladores, al identificar las ventajas y desventajas de varios patrones clave en contextos específicos. Al no existir un patrón superior, cada uno cumple un propósito particular, lo que permite a los desarrolladores seleccionar el más adecuado según las necesidades de su proyecto. Esta investigación no solo contribuye al ámbito académico, sino que también ofrece lecciones prácticas que pueden mejorar las prácticas de programación y facilitar la creación de software más eficiente y sostenible. La capacidad de elegir el patrón correcto no solo influye en la calidad del software, sino que también impacta en la eficiencia del proceso de desarrollo.
En conjunto, estos artículos ofrecen una visión integral sobre cómo la arquitectura de software y los patrones de diseño son esenciales para el desarrollo de software moderno. Al explorar diferentes enfoques y aplicaciones, se evidencia que la implementación de prácticas adecuadas no solo mejora la calidad del producto final, sino que también optimiza el proceso de desarrollo, preparando a los profesionales para enfrentar los retos del entorno tecnológico actual. La comprensión y aplicación de estos conceptos no solo es vital para el éxito de los proyectos de software, sino que también contribuye a la evolución de la industria, promoviendo una cultura de innovación y excelencia en el desarrollo de tecnologías. La interconexión de estos temas resalta la necesidad de un enfoque holístico que integre la educación, la práctica y la investigación, asegurando que los profesionales del software estén equipados para enfrentar las demandas del futuro.

\section{Plantiamiento del problema}
La complejidad creciente de las aplicaciones web y los entornos de computación en la nube exige enfoques sistemáticos y estructurados para garantizar la calidad, escalabilidad y mantenibilidad de los sistemas de software. Sin embargo, la falta de conocimiento y experiencia en patrones de diseño y arquitectura de software entre los desarrolladores puede limitar la capacidad de crear sistemas robustos y eficientes.

\section{Revisión de la Literatura}
La literatura actual refleja un interés creciente en integrar la
usabilidad desde las etapas iniciales del diseño de software
\cite{moreno2003}. La arquitectura de software, entendida como la
disposición de sus componentes, es crucial para mantener un bajo
acoplamiento y una alta cohesión \cite{romero2006}. Además, se observa
la imperiosa necesidad de adaptarse a cambios constantes en los
requisitos empresariales \cite{cristia2008}, lo que resalta la
importancia de patrones de diseño flexibles y escalables.

\section{Metodología}
Arquitectura Orientada a Servicios (SOA): Se enfoca en diseñar servicios autónomos que se comuniquen de manera eficiente, permitiendo una integración y comunicación más flexibles entre sistemas, modelado y Verificación de Patrones de Diseño: Utiliza metamodelos y herramientas gráficas para instanciar componentes arquitectónicos, lo que facilita el trabajo de los arquitectos de software y asegura la calidad del software, desarrollo de Aplicaciones Web con Patrones de Diseño: Aplica patrones de diseño para mejorar la usabilidad, el mantenimiento y la escalabilidad de las aplicaciones web, uso de Generadores de Código Fuente: Automatiza tareas repetitivas, reduce errores y optimiza el proceso de desarrollo de software, revisión Sistemática y Comparación de Patrones de Diseño: Identifica las ventajas y desventajas de cada patrón, permitiendo a los desarrolladores seleccionar el más adecuado según las necesidades del proyecto, integración de Patrones de Usabilidad en el Desarrollo de Software: Anticipa las necesidades del usuario, minimiza cambios necesarios en etapas posteriores y mejora la experiencia del usuario, arquitectura de Software para Robots Móviles: Aplica patrones de diseño para optimizar el control y la operación de robots móviles, utilizando sensores y arquitecturas deliberativas y reactivas.

\section{Objetivos:}
Análisis de la importancia de la arquitectura de software y los patrones de diseño: Investigar y analizar la importancia de la arquitectura de software y los patrones de diseño en el desarrollo de sistemas de software robustos, escalables y mantenibles, dentificación de las mejores prácticas para la aplicación de patrones de diseño: Identificar y analizar las mejores prácticas para la aplicación de patrones de diseño en diferentes contextos, como la computación en la nube, el desarrollo de aplicaciones web y la robótica, análisis de la importancia de la usabilidad y la experiencia del usuario: Investigar y analizar la importancia de la usabilidad y la experiencia del usuario en el desarrollo de software, y cómo se pueden integrar patrones de usabilidad en el desarrollo de software, investigación de los desafíos y limitaciones en la adopción de patrones de diseño: Investigar y analizar los desafíos y limitaciones en la adopción de patrones de diseño y arquitectura de software en el desarrollo de sistemas de software.

\section{Justificación:}
La arquitectura de software y los patrones de diseño son fundamentales en el desarrollo de sistemas de software robustos, escalables y mantenibles. Sin embargo, la falta de conocimiento y experiencia en patrones de diseño y arquitectura de software entre los desarrolladores puede limitar la capacidad de crear sistemas eficientes y de alta calidad.
Además, la complejidad creciente de las aplicaciones web y los entornos de computación en la nube exige enfoques sistemáticos y estructurados para garantizar la calidad, escalabilidad y mantenibilidad de los sistemas de software.
Por lo tanto, es fundamental investigar y analizar la importancia de la arquitectura de software y los patrones de diseño en el desarrollo de sistemas de software, y cómo se pueden aplicar en diferentes contextos para mejorar la calidad y la eficiencia de los sistemas.

\section{Conceptos:}
Arquitectura de software: Se refiere a la estructura y organización de los componentes de un sistema de software, incluyendo la relación entre ellos y con el entorno, patrones de diseño: Son soluciones reutilizables para problemas comunes en el desarrollo de software, que proporcionan una forma estructurada de abordar la complejidad del software, usabilidad: Se refiere a la facilidad con la que un usuario puede utilizar un sistema de software para realizar tareas específicas, experiencia del usuario: Se refiere a la percepción y satisfacción del usuario al interactuar con un sistema de software.

\section{Marco Teorico:}
Teoría de la arquitectura de software: Esta teoría se enfoca en la estructura y organización de los componentes de un sistema de software, y cómo éstos se relacionan entre sí y con el entorno, teoría de los patrones de diseño: Esta teoría se enfoca en la identificación y descripción de soluciones reutilizables para problemas comunes en el desarrollo de software, teoría de la usabilidad: Esta teoría se enfoca en la facilidad con la que un usuario puede utilizar un sistema de software para realizar tareas específicas.

\section{Estudios previos:}
Estudio de la arquitectura de software en la industria: Un estudio realizado por la IEEE encontró que la arquitectura de software es un factor clave en la determinación de la calidad y la mantenibilidad de los sistemas de software.
Estudio de los patrones de diseño en el desarrollo de software: Un estudio realizado por la ACM encontró que los patrones de diseño son una herramienta efectiva para mejorar la calidad y la productividad del desarrollo de software.
Estudio de la usabilidad en el desarrollo de software: Un estudio realizado por la Usability Professionals' Association encontró que la usabilidad es un factor clave en la determinación de la satisfacción del usuario y la adopción de los sistemas de software.

\section{Modelos y frameworks:}
Modelo de la arquitectura de software: El modelo de la arquitectura de software se enfoca en la estructura y organización de los componentes de un sistema de software.
Framework de los patrones de diseño: El framework de los patrones de diseño se enfoca en la identificación y descripción de soluciones reutilizables para problemas comunes en el desarrollo de software.
Framework de la usabilidad: El framework de la usabilidad se enfoca en la facilidad con la que un usuario puede utilizar un sistema de software para realizar tareas específicas.


\section{Resultados}
\vspace{0.5cm}

\begin{table}[h!]
\centering
\begin{tabular}{|p{1.2cm}|p{2.5cm}|p{2.5cm}|}
\hline
\textbf{Resultado} & \textbf{Descripción} & \textbf{Beneficio} \\ \hline
1. Mejora calidad & Patrones de diseño y arquitectura de software mejoran la calidad. & Reducción de errores y mejoras en la escalabilidad. \\ \hline
2. Optimización desarrollo & Generadores de código y automatización optimizan el desarrollo. & Reducción del tiempo y el esfuerzo. \\ \hline
3. Mejora usabilidad & Patrones de usabilidad mejoran la experiencia del usuario. & Mayor satisfacción y adopción. \\ \hline
4. Mayor eficiencia & Arquitectura orientada a servicios (SOA) mejora la comunicación. & Mayor flexibilidad y eficiencia. \\ \hline
5. Reducción complejidad & Patrones de diseño reducen la complejidad en aplicaciones web. & Mayor facilidad en el desarrollo. \\ \hline
6. Mejora robótica & Patrones de diseño y arquitectura de software mejoran la eficiencia y precisión. & Mayor productividad y reducción de errores. \\ \hline
\end{tabular}
\caption{Resultados de la aplicación de patrones de diseño y arquitectura de software}
\label{tabla:resultados}
\end{table}

\vspace{0.5cm}

\section{Conclusión:}
La aplicación de patrones de diseño y arquitectura de software es una práctica efectiva para mejorar la calidad, escalabilidad y mantenibilidad de los sistemas de software. Los resultados obtenidos en este estudio muestran que la aplicación de patrones de diseño y arquitectura de software puede mejorar la calidad del software, optimizar el proceso de desarrollo, mejorar la usabilidad y la experiencia del usuario, y aumentar la eficiencia en la comunicación entre sistemas. Por lo tanto, se recomienda que los desarrolladores de software consideren la aplicación de patrones de diseño y arquitectura de software en sus proyectos para mejorar la calidad y la eficiencia del desarrollo de software.

\section{Referencias:}
\bibitem{Blas2019}
Blas, M. J., Leone, H., \& Gonnet, S. (2019). Modelado y verificación de patrones de diseño de arquitectura de software para entornos de computación en la nube. \textit{RISTI - Revista Ibérica de Sistemas e Tecnologias de Informação}, 35, 1-17. \url{https://doi.org/10.17013/risti.35.1-17}

\bibitem{Ferrandis2021}
Ferrandis, (2021). Desarrollo de una herramienta para el aprendizaje de patrones de diseño software. \textit{RIUNET}. \url{https://riunet.upv.es/bitstream/handle/10251/174122/Ferrandis%20-%20Desarrollo%20de%20una%20herramienta%20para%20el%20aprendizaje%20de%20patrones%20de%20diseno%20software.pdf?sequence=1&isAllowed=y}

\bibitem{USS2020}
Universidad Señor de Sipán (USS). (2020). Patrones de diseño de software aplicado a las aplicaciones web. \textit{Repositorio USS}. \url{https://repositorio.uss.edu.pe/handle/20.500.12802/6783}

\bibitem{Scielo2018}
Scielo. (2018). Desarrollo de una arquitectura de software para el robot móvil Lázaro. \textit{Revista de Ingeniería}. \url{https://www.scielo.cl/scielo.php?pid=S0718-33052018000300376&script=sci_arttext}

\bibitem{Uniroja2006}
Uniroja. (2006). Arquitectura de software, esquemas y servicios. \textit{Revista Dialnet}. \url{https://dialnet.unirioja.es/servlet/articulo?codigo=4786655}

\bibitem{Redalyc2022}
Redalyc.org. (2022). Arquitectura de software para el desarrollo de herramienta tecnológica de costos, presupuestos y programación de obra. \textit{Revista Tecnológica}. \url{https://www.redalyc.org/journal/5337/533774788007/533774788007.pdf}

\bibitem{Redalyc2014}
Redalyc.org. (2014). Lenguajes de patrones de arquitectura de software: Una aproximación al estado del arte. \textit{Redalyc}. \url{https://www.redalyc.org/pdf/849/84933912003.pdf}

\bibitem{UNLP2022}
Universidad Nacional de La Plata (UNLP). (2022). Implementación de una arquitectura de software guiada por el dominio. \textit{Repositorio UNLP}. \url{https://sedici.unlp.edu.ar/bitstream/handle/10915/115198/Documento_completo.pdf-PDFA.pdf?sequence=1&isAllowed=y}

\bibitem{Repositorio2019}
(2019). Arquitectura de software basada en microservicios para desarrollo de aplicaciones web. \textit{Repositorio Institucional}. \url{https://acortar.link/6oUGHO}

\bibitem{Danilo2022}
Danilo, O., Patricia, N., \& Vinicio, M. (2022). Análisis comparativo de patrones de diseño de software. \textit{Polo del Conocimiento: Revista Científico-Profesional}, 7(7), 2146–2165. \url{https://dialnet.unirioja.es/descarga/articulo/9042927.pdf}

\bibitem{Scielo2021}
Scielo. (2021). Patrones de diseño GOF (The Gang of Four) en el contexto de procesos de desarrollo de aplicaciones orientadas a la web. \textit{Revista INFOTEC}. \url{https://www.scielo.cl/pdf/infotec/v24n3/art12.pdf}

\bibitem{Moreno2024}
Moreno, A. (2024). Patrones de usabilidad: Mejora de la usabilidad del software desde el momento arquitectónico. \textit{ResearchGate}. \url{https://www.researchgate.net/publication/221595496_Patrones_de_Usabilidad}

\bibitem{ProQuest2024}
ProQuest. (2024). Revisión sistemática sobre generadores de código fuente y patrones de arquitectura. \textit{ProQuest}. \url{https://acortar.link/CYBZ}

\section{Agradecimientos}

Agradezco sinceramente a mis instructores, quienes con su guía y apoyo incondicional, me permitieron llevar a cabo este trabajo de manera exitosa.

En particular, quiero expresar mi gratitud a:

\begin{itemize}
\item Carlos Julio Cadena Sarasty
\item Jhon Willian Corredor Araujo
\item Carlos Fabian Martinez Mora
\item Jesús Ariel Bonilla
\end{itemize}

Su dedicación, experiencia y conocimiento en el campo me fueron de gran ayuda y me permitieron crecer como profesional. Su apoyo y motivación me permitieron superar los desafíos y alcanzar mis objetivos.

Muchas gracias por su tiempo, esfuerzo y dedicación.
\end{document}